\documentclass[a4paper]{article}
\usepackage[utf8]{inputenc}
\usepackage[T2A]{fontenc}
\setlength{\textheight}{26.5cm}
\setlength{\textwidth}{18cm}
\setlength{\topmargin}{-25mm}
\setlength{\hoffset}{-25mm}
\def\zn{,\kern-0.09em,}

\begin{document}
\thispagestyle{empty}

\begin{flushleft}
Математички факултет\\
Универзитета у Београду
\end{flushleft}

\bigskip

\begin{center}
\textbf{МОЛБА\\
ЗА ОДОБРАВАЊЕ ТЕМЕ МАСТЕР РАДА
}\end{center}

\bigskip

\begin{flushleft}
Молим да ми се одобри израда мастер рада под насловом:
\end{flushleft}

\begin{minipage}{16.5cm}
%%%%%%%%%%%%%%%%%%%%%%%%%%%%%%%%%%%%%%%%%%%%%%%%%%%%%%%%%%%%%%%%%%%%%%%%%%%%%%%
% U donji red upisati naziv master rada umesto teksta: >>Назив мастер рада<<  %
%%%%%%%%%%%%%%%%%%%%%%%%%%%%%%%%%%%%%%%%%%%%%%%%%%%%%%%%%%%%%%%%%%%%%%%%%%%%%%%
\textbf{\textit{\zn Предвиђање функција протеина методама бинарне класификације'}}
\end{minipage}\\
\rule[4mm]{17.5cm}{.05mm}
\begin{flushleft}
\framebox{
\begin{minipage}[t][12cm]{17cm}
%%%%%%%%%%%%%%%%%%%%%%%%%%%%%%%%%%%%%%%%%%%%%%%%%%%%%%%%%%%%%%%%%%%%%%%%%%%%%%%
% 	-- unutrasnjost pravougaonika --    	  								  %
%%%%%%%%%%%%%%%%%%%%%%%%%%%%%%%%%%%%%%%%%%%%%%%%%%%%%%%%%%%%%%%%%%%%%%%%%%%%%%%
\textbf{Значај теме и области:}

% 	Umesto donjeg teksta opisati značaj teme i oblasti	%
Протеини су макромолекули који играју важну улогу у организму сваког живог бића. То су биолошки најактивнији молекули неопходни за изградњу и функционисање ћелија и имају велики број есенцијалних функција. Структура протеина зависи од распореда аминокиселина и утиче на његову функциjу. Примарна структура обухвата секвенцу аминокиселина које изграђују протеин. Она је основни извор информација о протеину и његовој функцији. Иако је познавање секвенце прилично корисно у многим пољима, као што су филогенетика и еволуциона биологија, за разумевање свих процеса који се дешавају у ћелији потребно је знање и о функцији протеина. Најчешћи начин за представљање функције протеина је преко система Gene Ontology. Овај систем дели функције протеина на три онтологије: биолошки процеси, молекулске функције и ћелијске компоненте. Свака онтологија представља усмерени ациклички граф где су чворовима придружени називи функција, и то сваком чвору специфичнија функција од функције његовог родитељског чвора. На тај начин, укупна функција једног протеина преставља подграф онтологије који садржи све чворове означене појединачним функцијама које протеин обавља. Познато је да протеини са сличним примарним структурама теже да обављају исте функције.
\\

\textbf{Специфични циљ рада:}

% 	Umesto donjeg teksta opisati specifični cilj master rada %
Сваке године секвенцира се велики број нових генома чиме расте број новооткривених протеина. Функција протеина одређује се експериментално што је скуп и спор процес због чега се улаже труд у развој рачунарских метода које могу да предвиде функцију протеина. Циљ овог рада је развој алата за одређивање функције протеина на основу његове примарне структуре помоћу метода бинарне класификације као што су Метод потпорних вектора, Случајне шуме и Логистичка регресија обучених на скупу протеина са експериментално утврђеним функцијама. За сваки чвор из онтологије ће на основу добијених предикција бити изграђен подграф онтологије који представља укупну функцију протеина.
\\

\textbf{Остале битне информације:}

% 	Umesto donjeg teksta навести друге битне информације %
За израду софтвера биће коришћен програмски језик Python. Биће коришћене имплементације наведених метода из библиотеке sklearn. Онтологија функција преузета је са адресе http://geneontology.org/docs/download-ontology/ у .obo формату, а база протеина са експериментално утврђеним функцијама са адресе  https://biofunctionprediction.org/cafa-targets/CAFA3\_training\_data.tgz.
\\


\end{minipage}
}
\end{flushleft}
\vspace{1cm}
%%%%%%%%%%%%%%%%%%%%%%%%%%%%%%%%%%%%%%%%%%%%%%%%%%%%%%%%%%%%%%%%%%%%%%%%%%%%%%%
% u donji red uneti:       ime i prezime, broj indeksa i modul studenta       %
%%%%%%%%%%%%%%%%%%%%%%%%%%%%%%%%%%%%%%%%%%%%%%%%%%%%%%%%%%%%%%%%%%%%%%%%%%%%%%%
\makebox[10cm][c]{\textbf{Ања Букуров, 1082/2016, информатика}}
%%%%%%%%%%%%%%%%%%%%%%%%%%%%%%%%%%%%%%%%%%%%%%%%%%%%%%%%%%%%%%%%%%%%%%%%%%%%%%%
% u donji red uneti:                   ime i prezime mentora				  %
%%%%%%%%%%%%%%%%%%%%%%%%%%%%%%%%%%%%%%%%%%%%%%%%%%%%%%%%%%%%%%%%%%%%%%%%%%%%%%%
Сагласан ментор \makebox[5cm][c]{\textbf{др Јована Ковачевић}} \\
\rule[4mm]{10cm}{.05mm} \hfill \raisebox{4mm}{\makebox[5cm][l]{.\dotfill.}} \\
\raisebox{1cm}%
[9mm][0mm]{\makebox[10cm][c]{\textit{(име и презиме студента, бр. индекса, модул)}}} \\
\makebox[10cm]{ }\\
\vspace{-1cm}\\
\rule[2cm]{6.5cm}{.05mm} \hfill \rule[2cm]{6.5cm}{.05mm}\\
\vspace{-2.4cm}\\
\raisebox{2cm}{\makebox[6.5cm][c]{\textit{(својеручни потпис студента)}}}
\hfill \raisebox{2cm}{\makebox[6.5cm][c]{\textit{(својеручни потпис ментора)}}}\\
\vspace{-2cm}\\
%%%%%%%%%%%%%%%%%%%%%%%%%%%%%%%%%%%%%%%%%%%%%%%%%%%%%%%%%%%%%%%%%%%%%%%%%%%%%%%
% u donji red uneti datum podnosenja molbe									  %
%%%%%%%%%%%%%%%%%%%%%%%%%%%%%%%%%%%%%%%%%%%%%%%%%%%%%%%%%%%%%%%%%%%%%%%%%%%%%%%
\makebox[5.5cm][c]{\textbf{<датум>}}\makebox[5.5cm]{}  Чланови комисије\\
%%%%%%%%%%%%%%%%%%%%%%%%%%%%%%%%%%%%%%%%%%%%%%%%%%%%%%%%%%%%%%%%%%%%%%%%%%%%%%%
% POPUNJAVA MENTOR (rucno ili na sledeci nacin):							  %
% u donji red umesto .\dotfill. upisati podatke o 1. clanu komisije		      %
%%%%%%%%%%%%%%%%%%%%%%%%%%%%%%%%%%%%%%%%%%%%%%%%%%%%%%%%%%%%%%%%%%%%%%%%%%%%%%%
\rule[4mm]{5.5cm}{.05mm}\makebox[5.5cm]{ } 1. \makebox[6cm][l]{.\dotfill.}\\
\vspace{-8mm}\\
\raisebox{4mm}%														
[7mm][0mm]{\makebox[5.5cm][c]{\textit{(датум подношења молбе)}}}\makebox[5.5cm]{ }
%%%%%%%%%%%%%%%%%%%%%%%%%%%%%%%%%%%%%%%%%%%%%%%%%%%%%%%%%%%%%%%%%%%%%%%%%%%%%%%
% POPUNJAVA MENTOR (rucno ili na sledeci nacin): 							  %
% u donji red umesto .\dotfill. upisati podatke o 2. clanu komisije           %
%%%%%%%%%%%%%%%%%%%%%%%%%%%%%%%%%%%%%%%%%%%%%%%%%%%%%%%%%%%%%%%%%%%%%%%%%%%%%%%
2. \makebox[6cm][l]{.\dotfill.}\\

\vspace{1cm}


\begin{flushleft}
%%%%%%%%%%%%%%%%%%%%%%%%%%%%%%%%%%%%%%%%%%%%%%%%%%%%%%%%%%%%%%%%%%%%%%%%%%%%%%%
% u donji red upisati              katedru									  %
%%%%%%%%%%%%%%%%%%%%%%%%%%%%%%%%%%%%%%%%%%%%%%%%%%%%%%%%%%%%%%%%%%%%%%%%%%%%%%%
Катедра \makebox[9.5cm][l]{\textbf{за рачунарство и информатику}} је сагласна са предложеном темом.
\vspace{-3mm}
\hspace*{13mm} \rule[2.3cm]{9.5cm}{.05mm}\\
\vspace{-1cm}
%%%%%%%%%%%%%%%%%%%%%%%%%%%%%%%%%%%%%%%%%%%%%%%%%%%%%%%%%%%%%%%%%%%%%%%%%%%%%%
% POPUNJAVA SEF KATEDRE                                                      %
%%%%%%%%%%%%%%%%%%%%%%%%%%%%%%%%%%%%%%%%%%%%%%%%%%%%%%%%%%%%%%%%%%%%%%%%%%%%%%
\makebox[6.5cm][c]{} \hfill \makebox[6.5cm][c]{}\\
\rule[4mm]{6.5cm}{.05mm} \hfill \rule[4mm]{6.5cm}{.05mm}\\
\vspace{-5mm}
\makebox[6.5cm][c]{\textit{(шеф катедре)}} \hfill \makebox[6.5cm][c]{\textit{(датум одобравања молбе)}}
\end{flushleft}
\end{document} 