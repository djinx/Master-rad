\chapter{Zaključak} % Main chapter title
\label{Chapter7}

U ovom radu prikazan je razvoj tri prediktora za predviđanje funkcije proteina. Računarski metodi za određivanje funkcija razvijaju se godinama, ali i dalje nema metoda koji može da odredi funkciju proteina preciznije od eksperimentalnog metoda. Kao što je već pomenuto, eksperimentalno utvrđivanje funkcije proteina je skup i spor proces zbog čega je ovaj problem i dalje aktuelan i od velikog značaja. 


Iako su se obučeni prediktori pokazali bolje od naivnog klasifikatora, nemaju približnu moć predviđanja u poređenju sa aktivnim rezultatima prikazanim na poslednjem CAFA takmičenju, najrelevantnijem takmičenju u ovoj oblasti. Planovi za unapređivanje prediktora obuhvataju:

\begin{itemize}
	\item treniranje pojedinačnih modela i prediktora za svaki od organizama - cilj je utvrditi da li će se prediktori bolje ponašati za određeni organizam ukoliko se obučavaju na proteinima koji potiču isključivo iz tog organizma,
	
	\item povećanje trening skupa - zbog malog broja pozitivnih instanci, za određene funkcije nije pravljen klasifikator,
	
	\item promenu ulaznih podataka - ideja je povećati vrednost parametra $k$ koji određuje dužinu podniski, a samim tim i niza kojim se predstavlja protein,
	
	\item korišćenje raznovrsnijih metoda binarne klasifikacije - na primer, neuronskih mreža.
\end{itemize}

