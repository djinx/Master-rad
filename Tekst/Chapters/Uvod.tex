\chapter{Uvod} % Main chapter title
\label{Chapter1}

Funkcionalna anotacija proteina može pomoći u dizajnu lekova za savremene bolesti jer mnoge od njih nastaju kao posledica izmene funkcije proteina usled mutacija \cite{review}. Zbog toga, predviđanje funkcije proteina predstavlja jedan od najbitnijih zadataka u bioinformatici. 

Metode za eksperimentalno određivanje funkcije proteina su spore u odnosu na brzinu sekvencioniranja genoma koje uvećava broj novih sekvenci. Do sada je za veoma mali broj proteina eksperimentalno određena funkcija zbog cene i trajanja tog procesa. Zbog toga se radi na razvijanju i unapređenju računarskih metoda za određivanje funkcije proteina. Proteini mogu imati više funkcija što omogućava sagledavanje problema predviđanja funkcije proteina kao problema višestruke klasifikacije.

Brzi razvoj računarskih metoda za predviđanje funkcije proteina doveo je do potrebe za njihovim poređenjem nad proteinima sa novoutvrđenim funkcijama. Zbog toga je kreiran eksperiment \textit{Critical Assessment of Function Annotation} (CAFA) \cite{biofCafa} koji se održava svake dve godine i gde autori prediktora šalju rezultate za veliki skup proteina za koje je funkcija nepoznata, pri čemu se za deo tog skupa funkcija određuje eksperimentalno pre evaluacije rezultata. Različiti algoritmi se evaluiraju prema sposobnosti da predvide koje funkcije obavljaju proteini. Preko 90\% metoda poređenih u CAFA takmičenju koriste informacije o sekvenci proteina na neki način \cite{jiang}. Jedan pristup je računanje sličnosti proteinskih sekvenci \cite{cozzeto, gong, lan} ili nekih drugih osobina sekvence, poput učestalosti pojavljivanja $k$-grama \cite{cozzeto} i obogaćivanje određenih podsekvenci u proteinima koje obavljaju određenu funkciju \cite{caoCheng}.

U ovom radu prikazan je razvoj alata za predviđanje funkcije proteina na osnovu njihove primarne strukture. Korišćene su metode binarne klasifikacije i to: metod potpornih vektora, logistička regresija i slučajne šume. Alat je razvijan u programskom jeziku Python. 

U poglavlju \ref{Chapter2} uvedeni su biološki pojmovi neophodni za razumevanje rada. U poglavlju \ref{Chapter3} prikazan je način predstavljanja bioloških podataka u računaru, a onda su ukratko predstavljene metode binarne klasifikacije korišćene za razvoj prediktora. Zatim, u poglavlju \ref{Chapter5}, analizirani su podaci o proteinima i njihovim funkcijama iz CAFA preporučenog trening skupa, nakon čega je opisana  implementacija alata za predviđanje funkcije proteina. Na kraju, u poglavlju \ref{Chapter6} sumirani su rezultati koje su prediktori dali na izdvojenom skupu proteina za testiranje, a zatim i poređenje sa aktuelnim rezultatima CAFA takmičenja.