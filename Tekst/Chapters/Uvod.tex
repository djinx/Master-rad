\chapter{Uvod} % Main chapter title
\label{Chapter1}

%  Lee2007...
%a0236.....
%bti1007
% jk i radivojac
Predviđanje funkcije proteina je jedan od najbitnijih zadataka bioinformatike koji može pomoći u velikom broju bioloških problema. Poznavanje funkcije proteina daje nam informacije o njegovim ulogama u organizmu. Metode za eksperimentalno određivanje funkcije proteina su spore u odnosu na brzinu sekvencioniranja genoma koje uvećava broj novih sekvenci. Mnoge metode predviđanja funkcije proteina zasnivaju se na poređenju sekvenci ili struktura proteina za koje je utvrđena funkcija sa onim proteinima za koje je funkcija nepoznata. 

U ovom radu prikazan je razvoj alata za predviđanje funkcije proteina na osnovu njihove primarne strukture. Korišćene su metode binarne klasifikacije i to: metod potpornih vektora, logistička regresija i slučajne šume. Alat je razvijan u programskom jeziku Python.

U poglavlju \ref{Chapter2} uvedeni su biološki pojmovi neophodni za razumevanje rada. U poglavlju \ref{Chapter3} prikazan je način predstavljanja bioloških podataka u računaru, a onda su ukratko predstavljene metode binarne klasifikacije korišćene za razvoj prediktora. Zatim, u poglavlju \ref{Chapter5}, analizirani su podaci o proteinima i njihovim funkcijama, nakon čega je opisana  implementacija alata za predviđanje funkcije proteina. Na kraju, u poglavlju \ref{Chapter6} sumirani su rezultati koje su prediktori dali na izdvojenom skupu proteina za testiranje.